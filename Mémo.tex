\documentclass{article}
\usepackage[T1]{fontenc}
\usepackage[utf8]{inputenc}
\usepackage{lmodern}
\usepackage[a4paper]{geometry}
\usepackage[]{amsmath}
\usepackage{xcolor}
\usepackage[]{amsfonts}
\usepackage{bbm}
\usepackage[]{mathtools}
\usepackage{hyperref}
\usepackage[]{cleveref}
\usepackage[]{amssymb}
\usepackage{amsthm}
\usepackage[bb=boondox]{mathalfa}
\DeclarePairedDelimiter\abs{\lvert}{\rvert}
\DeclarePairedDelimiter\norm{\lVert}{\rVert}
\usepackage[
backend=biber,
style=alphabetic,
sorting=ynt
]{biblatex}
\addbibresource{bib.bib}
\newtheorem{thm}{Theorem}
\crefname{thm}{theorem}{theorems}
\Crefname{thm}{Theorem}{Theorems}
\newtheorem{lem}{Lemma}
\crefname{lem}{lemma}{lemmas}
\Crefname{lem}{Lemma}{Lemmas}
\DeclareMathOperator{\Tr}{Tr}
\DeclareMathOperator{\disj}{disjoint}
\DeclareMathOperator{\tr}{tr}
\DeclareMathOperator{\Hess}{Hess}
\DeclareMathOperator{\Conn}{Conn}
\DeclareMathOperator{\supp}{supp}
\newcommand\formeq{\stackrel{\mathclap{\normalfont\mbox{F}}}{=}}
\begin{document}
\subsection{Intro}
Q.1: $\rho_\beta$ minimizes free energy.

\begin{eqnarray*}
   \rho_\beta:= \frac{e^{-\beta H}}{\Tr[e^{-\beta H}]}\\
F_\beta(\rho):=\Tr[\rho H]-\beta^{-1}S(\rho)\\
S(\rho):=-\Tr[\rho\log\rho]
\end{eqnarray*}
 We use the relative entropy $D(\rho\mid \sigma):=\Tr[\rho(\log \rho - \log  \sigma)]$ : it is nonnegative and zero iff $\rho=\sigma$.
 Then :
 $$
 D(\rho\mid\rho_\beta)=\Tr[\rho(\log\rho-\log\rho_\beta)]=-S(\rho)+\beta\Tr[\rho H]+\log(Z).
 $$
 where $Z=\Tr[e^{-\beta H}]$.
 
 Q.2: Among states with same energy ($\Tr[H\rho]$), $\rho_\beta$ maximizes entropy.
 We differentiate the entropy on the submanifold of quantum states with energy $E$.
 When $B$ is invertible, we have : $$
 dS(k)\vert_{\rho}=-\Tr[k\log \rho]-\Tr[k\rho\rho^{-1}]
 $$
 Because on the tangent space, $\Tr[k]=0$ and $\Tr[kH]=0$, we get that $$
 \Hess S(k)\vert_{\rho}=-\Tr[k^2\rho^{-1}]<0
 $$
 and $$
 dS(k)\vert_{\rho_\beta}=\beta\Tr[kH]=0.
 $$
 By concavity, $\rho_\beta$ is thus the only minimizer of $S$ with the same energy.
 
 \subsection{Proof of the bound given by \cite{froehlichPropertiesCorrelationsQuantum2015}}
 
 This whole section is based on two articles, \cite{froehlichPropertiesCorrelationsQuantum2015} and \cite{ueltschiClusterExpansionsCorrelation2005}.
 
 We work on $\Lambda\Subset \mathbb{Z}^d$, with the Hamiltonian $H_\Lambda=\sum_{X\subset \Lambda} \Phi_X$ where for $X$ in $\Lambda$, $\Phi_X$ is a bounded operator on $$\mathcal H_X:=\bigotimes_{x\in X}\mathcal H_x, \mathcal{H}_x\approx\mathbb C^N.$$
 
 We are interested in the correlation between observables residing in disjoint regions of $\Lambda$.
 An observable is a bounded operator on $\mathcal H_\Lambda$.
 We write $\supp A$ and call {\it support of $A$} the smallest set $X\subset \Lambda$ such that $$A=A’\otimes\bigotimes_{x\in \Lambda^C}\mathbbm 1,$$ where $A’$ is a bounded operator on $\mathcal H_X$.
 
 The expectation of an observable $A$ is defined as follows :
 $$
 \langle A \rangle := \frac{1}{Z_\Lambda}\tr Ae^{\beta H_\Lambda}
 $$
 where $\tr M:=\frac{\Tr M}{\dim(\mathcal H_\Lambda)}$ and $Z_\Lambda$, the {\it partition function}, is equal to $\tr e^{\beta H_\Lambda}$.
 
 With these definitions, we can state the following theorem, which shows that the correlation between regions of fixed size decreases exponentially with their distance in the following sense :
 \begin{thm}
     \label{thm:exp_decay_of_correlations}
     Let $b$ be a nonnegative function of the finite subsets of $\mathbb Z^d$.
     If we assume that 
     $$\beta\sup_{x\in\mathbb Z^d}\sum_{X\ni x}\lVert\Phi_X\rVert e^{\frac{3}{2}a|X|+b(X)}$$ 
     for some $a>0$, then for any two bounded operators $A,B$ on $\mathcal H_\Lambda$
     $$|\langle AB\rangle - \langle A\rangle\langle B\rangle|\leq k(A,B)e^{-\mu(\supp A,\supp B)}$$
     where $$k(A,B):=\lVert A\rVert\lVert B\rVert(a\abs{\supp A}+a\abs{\supp B} + 3a^2\abs{\supp A}\abs{\supp B})$$ and 
     $$
     \mu(X,Y):= \min_{\substack{X=X_0,X_1,\ldots,X_n=Y \\ \forall i=0,\ldots,n-1, X_i\cap X_{i+1}\neq \varnothing}}
     $$
 \end{thm}
    We will need a few results to prove this theorem.
     We begin with the {\it cluster expansion} of the partition function :
     \begin{align*}
         \tr e^{\beta H_\Lambda}=&\sum_{n>0}\frac{1}{n!}(\beta H_\Lambda)^n\\
         =&\sum_{n>0}\frac{\beta^n}{n!}\sum_{X_1,\ldots, X_n\subset \Lambda}\tr\Phi_{X_1}\ldots\Phi_{X_n}
     \end{align*}
     Observe that if $X$ and $Y$ are disjoint finite subsets of $\mathbb Z^d$, $\tr \Phi_X\Phi_Y=\tr \Phi_X \tr\Phi_Y$ and $\Phi_X$ and $\Phi_Y$ commute.
    This motivates the decomposition of the product into clusters.
    We see the finite subsets of $\mathbb Z^d$ as the vertices of a graph with an edge between $X$ and $Y$ if and only if $X\cap Y\neq \varnothing$; if it has a subgraph with vertices $X_1,\ldots,X_n$, we call $(X_1,\ldots,X_n)$ a {\it cluster}.
    We write $\mathcal C_\Lambda$ for the set of clusters over $\Lambda$.
    By summing over clusters, we get 
    \begin{align*}
        \tr e^{\beta H_\Lambda}=&\sum_{n\geq0}\frac {\beta^n} {n!} \sum_{k\geq 0}\frac{1}{k!}\sum_{\substack{C_1,\ldots,C_k\in\mathcal C_\Lambda \disj\\\abs{C_1}+\ldots+\abs{C_k}=n}}\binom{n}{\abs{C_1},\ldots,\abs{C_k}}\prod_{i=1,\ldots,k}\tr\prod_{X\in C_i}\Phi_X.
    \end{align*}
    The multinomial coefficient $\binom{n}{\abs{C_1},\ldots,\abs{C_k}}=\frac{n!}{\abs{C_1}!\ldots\abs{C_k}!}$ accounts for the order of the elements inside the clusters, and $\frac{1}{k!}$ for the order of the clusters.
    We can distribute the factorials and the $\beta^n$ : we write for a cluster $C=(X_1,\ldots,X_p)$
    $$
    w(C):=\frac {\beta^p}{p!}\tr \Phi_{X_1}\ldots\Phi_{X_p}
    $$
    and thus get 
    \begin{align*}
        \tr e^{\beta H_\Lambda}=&\sum_{n\geq0}\sum_{k\geq 0}\frac 1{k!}\sum_{\substack{C_1,\ldots,C_k\in\mathcal C_\Lambda \disj\\\abs{C_1}+\ldots+\abs{C_k}=n}}w(C_1)\ldots w(C_k)\\
        \formeq&\sum_{k\geq 0}\frac 1{k!} \sum_{\substack{C_1,\ldots,C_k\in\mathcal C_\Lambda\\\disj}}w(C_1)\ldots w(C_k).
    \end{align*}
    The $\formeq$ symbol is used to indicate that it’s a formal sum : we need absolute convergence to be able to swap the sums. 
    The proof of the absolute convergence is handled in \cite{friedliChapterClusterExpansion2017}, which allows us to remove the $\text{F}$.
    
    We can obtain a very similar formula for $\tr Ae^{\beta H_\Lambda}$ :
    \begin{equation*}
        \tr Ae^{\beta H_\Lambda}=\sum_{k\geq 0}\frac 1 {k!} \sum_{\substack{C_A,C_1,\ldots,C_k\\\disj}}w_A(C_A)w(C_1)\ldots w(C_k)
    \end{equation*}
    where $C_A$ is a cluster of the form $(\supp A,X_1,\ldots,X_p)$ and $$
        w_A(C_A):=\frac{\beta^n}{n!}\tr A\Phi_{X_1}\ldots\Phi_{X_n},
    $$
    and for $\tr ABe^{\beta H_\Lambda}$ :\begin{align*}
        \tr ABe^{\beta H_\Lambda}=&\sum_{k\geq 0}\frac 1 {k!} \sum_{\substack{C_A,C_B,C_1,\ldots,C_k\\\disj}}w_A(C_A)w_B(C_B)w(C_1)\ldots w(C_k)\\
        &+\sum_{k\geq 0}\frac 1 {k!} \sum_{\substack{C_{AB},C_1,\ldots,C_k\\\disj}}w_{AB}(C_{AB})w(C_1)\ldots w(C_k)
        \end{align*}
        where $C_{AB}$ is a cluster of the form $(\supp A, \supp B,X_1,\ldots,X_p)$ and
        $$
        w_{AB}(C_{AB}):=\frac{\beta^n}{n!}\tr AB\Phi_{X_1}\ldots\Phi_{X_n}.
        $$
        The two sums correspond to the cases in which $A$ and $B$ intersect respectively different clusters or the same cluster.
     
    In order to divide by $Z_\Lambda$, we introduce the following notation : for $M\subset \Lambda$ we write
    \begin{align*}
        Z_\Lambda(M)&:=\sum_{n\geq0}\frac 1{n!} \sum_{\substack{
            C_1,\ldots,C_n \disj\\
            \forall i=1,\ldots,n, \supp C_i\cap M =\varnothing
            }}w(C_1)\ldots w(C_n)
    \end{align*}
    where the support of a cluster $C=(X_1,\ldots,X_p)$ is $\supp C:=\supp X_1\cap\ldots\cap\supp X_p$.
    If $M=\varnothing$, we get $Z_\lambda$, which explains the notation.
    
    We can then write :
    \begin{align*}
        \langle A\rangle &= \frac 1 {Z_\Lambda}\sum_{C_A}w_A(C_A)\left[\sum_{k\geq 0} \frac 1{k!}\sum_{\substack{C_1,\ldots,C_k\disj\\\forall i=1,\ldots,k, \supp C_i\cap \supp C_A=\varnothing}}w(C_1)\ldots w(C_k)\right]\\
        &=\sum_{C_A}w_A(C_A)\frac{Z_\Lambda(\supp C_A)}{Z_\Lambda}
    \end{align*}
    and with the same decomposition :
    \begin{align*}
        \langle AB\rangle=& \sum_{\substack{C_{A},C_B\\\disj}}w_A(C_A)w_B(C_B)\frac{Z_\Lambda(\supp C_A \cup \supp C_B)}{Z_\Lambda}+\sum_{C_{AB}}w_{AB}(C_{AB})\frac{Z_\Lambda(\supp C_{AB})}{Z_\Lambda}.
    \end{align*}
    We can now use the following lemma :
    \begin{lem}
        \label{lem:Z/Z}
        Under the same hypotheses as in \cref{thm:exp_decay_of_correlations}, if $C_A$ and $C_B$ are disjoint,
        \begin{align*}
            \frac{Z_\Lambda(\supp C_A\cup \supp C_B)}{Z_\Lambda}&=\hat Z_\Lambda(C_A)\hat Z_\Lambda(C_B)+\hat Z_\Lambda(C_A,C_B)
        \end{align*}
        where \begin{align*}
            \hat Z_\Lambda(C_A)&:=\sum_{k\geq 0}(k+1)\sum_{C_1,\ldots,C_k}\varphi(C_A,C_1,\ldots,C_k)w(C_1)\ldots w(C_k)\\
            \hat Z_\Lambda(C_A,C_B)&:=\sum_{k\geq 0}(k+1)(k+2)\sum_{C_1,\ldots,C_k}\varphi(C_A,C_B,C_1,\ldots,C_k)w(C_1)\ldots w(C_k),\\
            \varphi(C_1,\ldots,C_n)&:=\begin{cases}
                1&\textnormal{if $n=1$,}\\
                \sum_{g\in \textnormal{Conn}(n)}\prod_{\{i,j\}\in g}(-\mathbbm 1_{\supp C_i\cap \supp C_j\neq \varnothing})&\textnormal{if $n\geq 2$.}
            \end{cases}
        \end{align*}
        $\textnormal{Conn}(n)$ denotes here the set of connected graphs over the vertices $\{1,\ldots,n\}$ and the product is taken over the edges of $g$.
    \end{lem}
        We will give the proof at the end because it utilizes convergences proved in later results.
    With the convention that $\Phi_\varnothing=\mathbb 0$ and $\tr \Phi_\varnothing=0$, this also gives us $$
        \frac{Z_\Lambda(\supp C_A)}{Z_\Lambda}=\frac{Z_\Lambda(\supp C_A\cup \supp (\varnothing))}{Z_\Lambda}=\hat Z_\Lambda(C_A)\times\mathbbm 1+ \mathbb 0=\hat Z_\Lambda(C_A),
    $$
    because $\varphi((\varnothing))=1$ and $\varphi((\varnothing),C_1,\ldots,C_k)=0$ for $k\geq 1$.

    
 Using this lemma, we get \begin{align*}
 \langle A\rangle\langle B \rangle =& \left(\sum_{C_A}w_A(C_A)\hat Z_\Lambda(C_A)\right)\left(\sum_{C_B}w_B(C_B)\hat Z_\Lambda(C_B)\right)\\
 =&\sum_{C_A,C_B}w_A(C_A)w_B(C_B)\hat Z_\Lambda(C_A)\hat Z_\Lambda(C_B) 
 \end{align*}
 and 
 \begin{align*}
     \langle AB\rangle =\sum_{\substack{C_A,C_B\\\disj}}w_A(C_A)w_B(C_B)\hat Z_\Lambda(C_A)\hat Z_\Lambda(C_B)+\sum_{\substack{C_A,C_B\\\disj}}w_A(C_A)w_B(C_B)\hat Z_\Lambda(C_A,C_B)&
     \\
     +\sum_{C_{AB}}w_{AB}(C_{AB})\hat Z_{\Lambda}(C_{AB}).&
 \end{align*}
 
 Because $\sum_{\substack{C_A,C_B\\\disj}}(\ldots)-\sum_{C_A,C_B}(\ldots)=-\sum_{\substack{C_A,C_B\\\supp C_A\cap \supp C_B\neq \varnothing}}(\ldots)$,
 we get the following expression for the covariance:
 \begin{align*}
     \langle AB\rangle - \langle A\rangle \langle B\rangle =\sum_{\substack{C_A,C_B\\\disj}}w_A(C_A)w_B(C_B)\hat Z_\Lambda(C_A,C_B)
     +\sum_{C_{AB}}w_{AB}(C_{AB})\hat Z_{\Lambda}(C_{AB})\\
     -\sum_{\substack{C_A,C_B\\\supp C_A\cap \supp C_B\neq \varnothing}}w_A(C_A)w_B(C_B)\hat Z_\Lambda(C_A)\hat Z_\Lambda(C_B).
\end{align*}

We will give the proof of this and other intermediary results at the end of this section.
The main result we will use is the following : 
\begin{thm}
    Assume that for every cluster $C$,$$
        \sum_{\substack{C’\in \mathcal C_\Lambda\\\supp C\cap \supp C’\neq\varnothing}}\abs{w(C’)}e^{a\abs{\supp C’}+b(C’)}\leq a\abs{\supp C},
    $$
    where if $C=(X_1,\ldots,X_n),b(C)=b(X_1)+\ldots+b(X_n)$.
    Then for every $m\geq 1$, $A_1,\ldots ,A_m$ clusters,
    \begin{multline*}
        \sum_{k\geq 0}\frac{(k+m)!}{k!}\sum_{C_1,\ldots,C_k}\abs{\varphi(A_1,\ldots,A_m,C_1,\ldots,C_k)}\abs{w(C_1)}\ldots \abs{w(C_k)}e^{b(C_1)+\ldots+b(C_k)}\\
            \leq e^{\frac{2^m-1}{m}a(\abs{\supp C_1}+\ldots+\abs{\supp C_k})}\times 2^{\#\{1\leq i<j\leq m\mid A_i\cap A_j\neq\varnothing\}}.
        \end{multline*}
\end{thm}
 Let us prove the hypothesis.
 For $C=(X_1,\ldots,X_n)$ a cluster,
 \begin{align*}
|w(C)|=&\frac{\beta^{n}}{n!}\abs{\tr\Phi_{X_1}\ldots\Phi_{X_n}}\\
\leq& \frac{\beta^{n}}{n!}\norm{\Phi_{X_1}\ldots\Phi_{X_n}}\\
\leq &\frac{\beta^{n}}{n!}\prod_{i=1}^n{\norm{\Phi_{X_i}}}.
 \end{align*}
 
 
 
 
 \begin{proof}[Proof of \cref{lem:Z/Z}]
     In the following, $\mathcal G(V)$ denotes the set of graphs with vertices $V$ and $\Conn(V)$ the set of connected graphs with vertices $V$.
     We also allow ourselves to write $C$ for $\supp C$ when taking intersections to lighten the notations -- by $C_i\cap C_j \neq \varnothing$ we never mean that $C_i$ and $C_j$ have an element in common, but that their supports intersect.
     Let us now rewrite $Z_\Lambda(\supp C_A\cup \supp C_B)$ to find the expression we want:
     \begin{equation}
         \label{eqn:ZCACB_with_product}
         Z_\Lambda(\supp C_A\cup \supp C_B)=\sum_{n\geq 0}\frac 1{n!}\sum_{C_1,\ldots,C_n}\left(\prod_{\substack{i,j\in \{1\ldots,n,A,B\}\\
                 i\neq j}}\mathbb 1_{C_i\cap C_j=\varnothing}\right)w(C_1)\ldots w(C_n)
     \end{equation}
     Note that we included the factor $\mathbb 1_{C_A\cap C_B=\varnothing}$ in the product, which we can since it is just equal to $1$.
     We then rewrite the product to get a sum over graphs:
     \begin{align*}
         \prod_{\substack{i,j\in \{1\ldots,n,A,B\}\\
                 i\neq j}}\mathbb 1_{C_i\cap C_j=\varnothing}=&\prod_{\substack{i,j\in \{1\ldots,n,A,B\}\\
                 i\neq j}}(1-\mathbb 1_{C_i\cap C_j\neq\varnothing})\\
         =&\sum_{g \in \mathcal G(\{1,\ldots,n,A,B\})}\prod_{\{i,j\}\in g}(-\mathbb 1_{C_i\cap C_j\neq \varnothing}).
     \end{align*}
     We can split the sum between the graphs where $A$ and $B$ are in the same connected component and those where they aren’t (in the graph $g$, this does not say anything about intersections of support of clusters).
     In the first case, we extract their connected component as a connected graph over $A$,$B$ and some subset $V$ of the vertices:
     \begin{align*}
         \sum_{\substack{g \in\mathcal G (\{1,\ldots,n,A,B\})\\A\leftrightarrow B}}\prod_{\{i,j\}\in g}(-\mathbb 1_{C_i\cap C_j\neq\varnothing})=\sum_{V\subset \{1,\ldots,n\}}\sum_{g_{A,B}\in\Conn(\{A,B\}\cup V)}\left[ \left(\prod_{\{i,j\}\in g_{AB}}(-\mathbb 1_{C_i\cap C_j\neq \varnothing})\right)\right. &\\
         \left. \sum_{g\in \mathcal G(\{1,\ldots,n\}\setminus V)}\prod_{\{i,j\}\in g}(-\mathbb 1_{C_i\cap C_j\neq\varnothing})\right]&\\
         =\sum_{V=\{v_1,\ldots,v_l\}\subset\{1,\ldots,n\}}\Biggl(\varphi(C_A,C_B,C_{v_1},\ldots,C_{v_l})
         \prod_{\substack{i,j\in \{1,\ldots,n\}\setminus V\\i\neq j}}\mathbbm 1_{C_i\cap C_j=\varnothing}\Biggr)&
     \end{align*}
     and in the second, we do the same but by extracting the connected component of $A$, with vertices $\{A\}\cup V$, and the connected component of $B$, with vertices $\{B\}\cup W$:   
     \begin{align*}
         \sum_{\substack{g \in\mathcal G (\{1,\ldots,n,A,B\})\\A\nleftrightarrow B}}\prod_{\{i,j\}\in g}(-\mathbb 1_{C_i\cap C_j\neq\varnothing})=\sum_{\substack{V=\{v_1,\ldots,v_l\}\subset \{1,\ldots,n\}\\
                 W=\{w_1,\ldots,w_m\}\subset\{1,\ldots,n\}\\
                 V\cap W=\varnothing}}\Biggl(\varphi(C_A,C_{v_1},\ldots,C_{v_l})\varphi(C_B,C_{w_1},\ldots,C_{w_m})\\
         \prod_{\substack{i,j\in \{1\ldots,n\}\setminus (V \cup W)\\i\neq j}}\mathbbm 1_{C_i\cap C_j=\varnothing}\Biggr).
     \end{align*}
     Now we "just" plug these expressions in \cref{eqn:ZCACB_with_product} and sum over the cardinals of $V$ and $W$:        \begin{align*}
         Z_\Lambda(\supp C_A\cup \supp C_B)=\sum_{n\geq 0}\frac 1{n!}\Biggl[\sum_{l\leq n} \frac{n!}{l!(n-l)!}\left(\sum_{C_1,\ldots,C_l}(l+2)!\varphi(C_A,C_B,C_1,\ldots,C_l)w(C_1)\ldots w(C_l)\right)
         \\
         \times\sum_{\substack{C_{l+1},\ldots C_n\\\disj}}w(C_{l+1},\ldots,C_n)
         \\
         +\sum_{\substack{l,m>0\\l+m\leq n}}\frac{n!}{m!l!(n-m-l)!}\left(\sum_{C_1,\ldots,C_l}(l+1)!\varphi(C_A,C_1,\ldots,C_l)w(C_1)\ldots w(C_l)\right)
         \\
         \times
         \left(\sum_{C_{l+1},\ldots,C_{l+m}}(m+1)!\varphi(C_B,C_{l+1},\ldots,C_{l+m})w(C_{l+1}\ldots w(C_{l+m})\right)
         \\
         \times
         \sum_{\substack{C_{l+m+1},\ldots,C_{n}\\\disj}}w(C_{l+m+1})\ldots w(C_n)\Biggr].
     \end{align*}
     The first two lines correspond to the sum over graphs with $A$ and $B$ connected and the following three to $A$ and $B$ disconnected. 
     The combinatorial terms outside of the parentheses are multinomial coefficients that account for the number of ways to pick $V$ and $W$. 
     I’m not sure what the factorials inside the parentheses are, but they definitely should be there.
     By distributing the factorials correctly and swapping the sums we get :
     $$
     \frac{Z_\Lambda(\supp C_A\cup \supp C_B)}{Z_\Lambda}\formeq\hat Z_\Lambda(C_A)\hat Z_\Lambda(C_B)+\hat Z_\Lambda(C_A,C_B).
     $$
     We will show that the sums converge absolutely and uniformly in $n$ {\color{red}later}, which will conclude the proof.
 \end{proof} 
 \printbibliography
\end{document}
